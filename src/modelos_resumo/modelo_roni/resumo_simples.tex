\documentclass{modelo_resumo_simples}

\primeiroAutor{Maria Torres}
\abreviacaoPrimeiroAutor{TORRES, M.}
\notaPrimeiroAutor{Estudante do curso de bacharelado em Sistemas de Informação, IF Goiano – Campus Ceres.}

\segundoAutor{Bob Camargo Silva}
\abreviacaoSegundoAutor{SILVA, B. C.}
\notaSegundoAutor{Estudante do curso Técnico em Meio Ambiente, IF Goiano – Campus Ceres.}

\terceiroAutor{Ana Maria Cunha}
\abreviacaoTerceiroAutor{CUNHA, A. M.}
\notaTerceiroAutor{Estudante do curso Técnico em Informática, IF Goiano – Campus Ceres.}

\quartoAutor{Pedro Alvares Cabral}
\abreviacaoQuartoAutor{CABRAL, P. A.}
\notaReferenciaQuartoAutor{3}

\quintoAutor{Joana da Silva Amaral}
\abreviacaoQuintoAutor{AMARAL, J. S.}
\notaReferenciaQuintoAutor{2}

\sextoAutor{Ronneesley Moura Teles}
\abreviacaoSextoAutor{TELES, R. M.}
\notaSextoAutor{Professor orientador, IF Goiano – Campus Ceres.}

\titulo{Titulo do trabalho}

\begin{document}

	\construirtitulo

	\construirautores
	
	\begin{resumo}
	Na \textbf{introdução} você deve apresentar o problema e sua importância. Em seguida, entra o trecho do texto relacionado aos \textbf{objetivos}. Você sempre deve apresentar pelo meno um objetivo que é o objetivo geral.
	Em seguida entra-se na seção de \textbf{materiais e métodos}. Nela, você deve apresentar as técnicas que foram utilizadas para resolver o problema. Lembrando que sempre o texto deve estar no passado. Por exemplo: foi utilizado, processou-se os dados, etc.
	A próxima seção é a seção dos \textbf{resultados e discussão}. Nela você deve apresentar os resultados encontrados abrindo o texto para as interpretações destes resultados assim como suas consequências ou impactos sociais, econômicos, políticos, etc.
	Por fim, entra-se a seção de \textbf{conclusão}. Nela você fará uma consideração final breve e sucinta com os principais produtos ou resultados do trabalho.
	\end{resumo}
	
	\begin{palavras_chave}
	Computação; Algoritmo; Programação Linear
	\end{palavras_chave}

\end{document}